\documentclass[compress]{beamer}
\usepackage[T1]{fontenc}

\beamertemplatenavigationsymbolsempty

%\usetheme{Madrid}

\usetheme[secheader]{Madrid}

\usepackage{mathtools,amssymb,amsthm,amsfonts,beamerthemesplit,amsmath}
\usepackage{enumerate,mathrsfs,subfig,array}

\usepackage[english]{babel}
\usepackage{csquotes}

\usepackage{graphicx}

\usepackage{color}

\usepackage{makecell}


\xdefinecolor{crimson}{RGB}{158,27,52}
\xdefinecolor{blue}{RGB}{0,124,194}
%\xdefinecolor{gray}{RGB}{0,0,0}
\xdefinecolor{gray}{RGB}{96,106,116}
\xdefinecolor{teal}{RGB}{0,177,176}
\xdefinecolor{yellow}{RGB}{235,215,34}
\xdefinecolor{darkBlue}{RGB}{12, 12, 112}

\usecolortheme[named=darkBlue]{structure}

%\usecolortheme[named=crimson]{structure}


\newtheorem{conjecture}[theorem]{Conjecture}




\setbeamertemplate{itemize items}[default]
\setbeamertemplate{enumerate items}[default]

%Shortcuts for symbol for R, C, etc
\newcommand{\N}{\mathbb{N}}
\newcommand{\Z}{\mathbb{Z}}
\newcommand{\R}{\mathbb{R}}
\newcommand{\C}{\mathbb{C}}
\newcommand{\Q}{\mathbb{Q}}
\newcommand{\F}{\mathbb{F}}
\newcommand{\E}{\mathbb{E}}
\newcommand{\T}{\mathbb{T}}
\newcommand{\K}{\mathbb{K}}
\newcommand{\X}{\mathbb{X}}
\newcommand{\Y}{\mathbb{Y}} 


\title{IOTO Team}

\subtitle{Python Challenge}

\author[D. Berman, J.Glassett, D. Liu, D. Taha, A.Zheng]{Dana Berman\\ Jillian Glassett\\ Danyi Liu\\ Diaaeldin Taha\\  Aaron (Xiang) Zheng}


\date{Math\^{}Industry, PIMS\\
August 2020}


% Delete this, if you do not want the table of contents to pop up at
% the beginning of each subsection:
\mode<presentation>{%
  %% At the begin of a section, insert a short outline
  \AtBeginSection[]{%
    \begin{frame}<beamer>%
      \frametitle{Outline}
      \tableofcontents[currentsection=show/shaded/hide]%
    \end{frame}%
  }%
  %% 
  %% Same for Subsections
 % \AtBeginSubsection[]{%
  %  \begin{frame}<beamer>%
  %    \frametitle{Outline}
  %    \tableofcontents[currentsection,subsectionstyle=show/shaded/hide]%
  %  \end{frame}}%
}

\setbeamertemplate{headline}
{%
  \leavevmode%
  \begin{beamercolorbox}[wd=.5\paperwidth,ht=2.5ex,dp=1.125ex]{section in head/foot}%
    \hbox to .5\paperwidth{\hfil\insertsectionhead\hspace{2pt}}
  \end{beamercolorbox}%
  \begin{beamercolorbox}[wd=.5\paperwidth,ht=2.5ex,dp=1.125ex]{subsection in head/foot}%
    \hbox to .5\paperwidth{\hspace{2pt}\insertsubsectionhead\hfil}
  \end{beamercolorbox}%
}

\setbeamertemplate{navigation symbols}{} 

% Let's get started
\begin{document}

\begin{frame}
  \titlepage
\end{frame}

\begin{frame}{Outline}
  \tableofcontents
  % You might wish to add the option [pausesections]
\end{frame}

% Section and subsections will appear in the presentation overview
% and table of contents.
\section{First Section}


\begin{frame}{First Slide Title}

\end{frame}

    



\subsection{First Observation}

% You can reveal the parts of a slide one at a time
% with the \pause command:
\begin{frame}{Second Slide Title}
 

\end{frame}

\section{Second Section}

\subsection{}



% Placing a * after \section means it will not show in the
% outline or table of contents.
\section*{Summary}

\begin{frame}{Summary}
  
\end{frame}



% All of the following is optional and typically not needed. 
\appendix
\section<presentation>*{\appendixname}
\subsection<presentation>*{Sources}

\begin{frame}[allowframebreaks]
  \frametitle<presentation>{Sources}
  
  

    
  \begin{thebibliography}{10}
    
  %\beamertemplatebookbibitems
  % Start with overview books.
  
  \bibitem{GC_dataset}
  {GC} Service Inventory
  \newblock \url{https://open.canada.ca/data/en/dataset/3ac0d080-6149-499a-8b06-7ce5f00ec56c}
  \newblock Accessed: 2020-8-13
  
  %Add git-hub repo?
  
  
  \end{thebibliography}
\end{frame}

\end{document}


